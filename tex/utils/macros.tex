% \newcommand{\jedit}[2]{\sout{#1}{\color{red}{#2}}}
\newcommand{\jedit}[2]{#2}
\newcommand{\jcomment}[1]{({\color{pgreen}{JRO's comment:}} \textbf{\color{pgreen}{#1}})}

\newcommand{\citationNeeded}{\textcolor{magenta}{\textbf{[CITATION NEEDED]}}\xspace}
\newcommand{\tableNeeded}{\textcolor{magenta}{\textbf{[TABLE NEEDED]}}\xspace}
\newcommand{\figureNeeded}{\textcolor{magenta}{\textbf{[FIGURE NEEDED]}}\xspace}
\newcommand{\highLight}[1]{\textcolor{magenta}{\MakeUppercase{#1}}}

\newcommand{\fig}{Figure\xspace}
\newcommand{\figs}{Figures\xspace}
\newcommand{\tbl}{Table\xspace}
\newcommand{\tbls}{Tables\xspace}

\newcommand{\datasets}{data sets\xspace}
\newcommand{\dataset}{data set\xspace}

\newcommand{\ignore}[1]{}
\newcommand{\addTail}[1]{\textit{#1}.---}
\newcommand{\super}[1]{\ensuremath{^{\textrm{#1}}}}
\newcommand{\sub}[1]{\ensuremath{_{\textrm{#1}}}}
\newcommand{\dC}{\ensuremath{^\circ{\textrm{C}}}}
\newcommand{\tn}{\tabularnewline}
\newcommand{\spp}[1]{\textit{#1}}

\providecommand{\e}[1]{\ensuremath{\times 10^{#1}}}

\newcommand{\change}[2]{{\color{red} #2}\xspace}
\newcommand{\thought}[1]{\textcolor{purple}{THOUGHT: #1}}

\newcommand{\widthFigure}[5]{\begin{figure}[htbp]
\begin{center}
    \includegraphics[width=#1\textwidth]{#2}
    \captionsetup{#3}
    \caption{#4}
    \label{#5}
    \end{center}
    \end{figure}}

\newcommand{\heightFigure}[5]{\begin{figure}[htbp]
\begin{center}
    \includegraphics[height=#1\textheight]{#2}
    \captionsetup{#3}
    \caption{#4}
    \label{#5}
    \end{center}
    \end{figure}}

\newcommand{\smartFigure}[5]{%
    \begin{figure}[htbp]
        \begin{center}
            \includegraphics[width=\textwidth,height=#1\textheight,keepaspectratio]{#2}
            \captionsetup{#3}
            \caption{#4}
            \label{#5}
        \end{center}
    \end{figure}
}

\newcommand{\mFigure}[4]{\smartFigure{#1}{#2}{listformat=figList}{#3}{#4}\clearpage}
\newcommand{\embedHeightFigure}[4]{\heightFigure{#1}{#2}{listformat=figList}{#3}{#4}}
\newcommand{\embedWidthFigure}[4]{\widthFigure{#1}{#2}{listformat=figList}{#3}{#4}}
\newcommand{\siFigure}[4]{\smartFigure{#1}{#2}{name=Figure S, labelformat=noSpace, listformat=sFigList}{#3}{#4}\clearpage}

%% macro to make long strings breakable over lines
\makeatletter
\def\breakable#1{\xHyphen@te#1$\unskip}
\def\xHyphen@te{\@ifnextchar${\@gobble}{\sw@p{\allowbreak{}\xHyphen@te}}}
% \def\xHyphen@te{\@ifnextchar${\@gobble}{\sw@p{\hskip 0pt plus 1pt\xHyphen@te}}}
\def\sw@p#1#2{#2#1}
\makeatother
