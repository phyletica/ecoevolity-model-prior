\subsection{Analyses of simulated data}

\subsubsection{Comparing three models of shared evolutionary events}

Summary Statistics
Before looking at the results here are some summary statistics that we need to 
understand in order to interpret the results. Coverage or credible sites is the probability
of the number events within the credible sites. The probability of estimated number 
of events equal to the true number of events which basicially tells us how often we 
are right. Median posterior probability is the true number of events given the data 
and this summary statistics tells us how much support we have for the correct answer.
Root mean square error quantifies how much error is there and comparing the true 
value to the estimate value. 
Hyper Parameter Estimation 
We are going to look how well the hyper parameters are estimated. Dirichlet
prior has as a single parameter called concentration parameter that tells how cluster 
the process is. The result of estimated concentration versus true concentration plot
shows that the true value is within 95\% credible sites. Pitman-Yor prior has two 
parameters(concentration parameter and discount parameter) the estimated value of 
concentration versus the true value indicates that the true is also within 95\%
credible sites. The Uniform prior has split weight and the result of estimated value 
versus the true value shows the true value is little less comparing to the other two 
priors but it's still within 94\% credible sites.
In this study, we are interested in divergence time and sharing divergence time and 
everything else is nuisance parameters. For both divergence time and sharing divergence time 
estimation we analyzed data with constant sites and without constant sites. First, we are going 
to look at the results of divegence time estimation with constant sites. The models which generated 
the data in rows and columns represent the models under which the data is analyzed. The true divergene 
time is on x-axis and the estimated divergence time is on y-axis. The results of true divergence 
time versus estimated divengence time show that all the three models (DP, PYP, and SW) performed
equally well with very similiar coverage around 95\%. Except for independent all others have truths 
contianed in within that 95\% credible interval. Independent is little lower because it's evolving 
indepdently and we are not modeling that way. We haven't observed any patterns that differiate
these 3 models. Root mean square errors are very similiar across all the models as well. 
With variable sites only there are more variation in estimates when throwing the invariable 
sites which contain lot of information and when we use them we have much more presicion in estimates
than ingoring them. All 10 pairs diverge spontatnous recently causing some MCMC issues and there 
are very few variable sites all 10 pairs diverged recently. Basicailly there is discrepency bewtween 
gene divergence and population divergence which is the largest at the most recent divergent times.
Comparing the true number of events with estimated number of events among all the sites. The true 
divergence time ranges from 1 to 10 and 1 they all share the divergence time and 10 they all 
diverged independently. The true number of events is the one has the higest posterior probability.
We have 520 replicates and the grid shows how many of them dall in each cell. The result shows 
that the data geenrated under PYP and analyzed under PYP. There are 88 datasets for which the true 
number of divergence is 5 and 76 out of those 88 got the that number correctly. 11 out of 88 
estimated 4 diverngence times and 1 out of 88 estimated 3 divergence time. 
The results for two constraint models all the 3 models under which the data is analyzed show that 
half of the time there are 10 indepedent divergence times for the datasets that is generated under 
indepdent model. The models are doing very well and they got half of time the correct asnwer. The 
ones got wrong are not that far away from the true and they are 8 or 9 event. PYP seems like 
doing little better in terms of how often ge tthe correct answer and also the confident in that 
correct asnwer. The median posterior probability for k=10 getting the correct answer
given the data is 47\% and it's little lower for the other models. The scenario of co-divergence
is an easy scenario and it has the proability of 1 for true number of event versus estimated number of 
event. When the true 1 the models are doing very well estimating the number of divergence time. But 
when the truth is 10 everything is diverging indepdently then it's more difficult scenario for 
modles to handle. Therefore, there is variation in the results. 
The results of true number of events versus estimated number of events among variable sites. The
constant sites are very informative and without invariable sites, the median posterior probability
which is the correct answer is much lower. But the covarage is unaffacted because the 
likelihood for the missing invariable characters is corrected in Ecoevolity. PYP is doing better when 
dealing with the most difficult scenario all the 10 pairs are diverging independetly as well. Both in
terms of covrage and getting the correct answer is much higher than the other two models. 

\subsubsection{Comparing DP, PYP, and \wunif across \dataset sizes}

% #### 
% Nuisane Parameters


\subsection{Empirical application?}
