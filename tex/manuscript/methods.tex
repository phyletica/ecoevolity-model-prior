\subsection{Analyses of simulated data}

We used \simcoevolity and \ecoevolity of the \ecoevolity software
package
\citep[Version 0.3.2 Commit c1685dfa][]{Oaks2018ecoevolity}
to simulate and analyze \datasets, respectively.
Each simulated \dataset comprised biallelic characters from 2 diploid
individuals (4 genomes) sampled per population from 10 pairs of populations
(i.e., 10 divergence comparisons).
We assumed the mutation rates were equal and 1 across the 10 pairs of
populations, such that time and effective population sizes are scaled by the
mutation rate.
The times of divergence events were exponentially distributed with a mean 0.05
expected subsitutions per site.
The mutation-scaled effective sizes ($\epopsize\murate$) of the descendant
populations were gamma-distributed with a shape of 20 and mean of 0.001,
and the size of the ancestral population relative to the mean of its
descendants was gamma distributed with a shape of 50 and mean of 1.
These distributions were also used as priors when the simulated \datasets were
analyzed with \ecoevolity.


When analyzing each simulated \dataset with \ecoevolity,
we ran four MCMC chains for 75,000 generations with a sample taken every 50
generations.
We summarized the last 1000 samples from each chain for a total of 4000 MCMC
samples to approximate the posterior distribution for every simulated \dataset.
From the four chains of each analysis, we calculated the potential
scale reduction factor \citep[PSRF; the square root of Equation 1.1
in][]{Brooks1998} and effective sample size \citep[ESS;][]{Gong2014} for all
continuous parameters and the log likelihood using
\pycoevolity (Version 0.2.6 Commit 27cb15e5).
When plotting results, we highlight any simulation replicates that have a
$\textrm{PSRF} > 1.2$.

\subsubsection{Comparing three models of shared evolutionary events}


We simulated 720 \datasets comprised of 500,000 biallelic characters under five
different models of how divergences are clustered across 10 pairs of
populations.
\begin{description}
    \item[Simultaneous] All 10 pairs of populations diverged at the same time
        (which was exponentially distributed).
    \item[Independent] All 10 pairs of populations diverged independently.
    \item[DP] Divergence times distributed according to a Dirichlet process
        where the concentration parameter is distributed as
        $\concentration \sim \dgamma{2}{5.42}$.
    \item[PYP] Divergence times distributed according to a Pitman-Yor process
        where the concentration and discount parameters are distributed as
        $\concentration \sim \dgamma{2}{3.58}$
        and
        $\discount \sim \dbeta{1}{4}$.
    \item[\wunif] Divergence times distributed according to a
        weighted-uniform distribution
        where the split-weight parameter is distributed as
        $\splitweight \sim \dgamma{0.55}{4.026}$.
\end{description}
The hyperpriors on the parameters of the DP, PYP, and \wunif
were selected such that the mean number of divergence events
is approximately 5.5
and the probability of each possible number of events
is similar among the three models.
% Each simulated \dataset comprised 500,000 unlinked biallelic characters from 10
% diploid individuals (20 genomes) sampled per population from three demographic
% comparisons.

\subsubsection{Comparing across \dataset sizes}
