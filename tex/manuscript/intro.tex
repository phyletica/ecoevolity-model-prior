Introduction here \ldots

In this study, we compared the performance of 3 models (Dirichlet prior, Pitman-Yor prior and Uniform prior) 
that are implemented in Ecoevolity software which is the primarly software we used to estimate the number of shared divergence time 
among 10 pairs of populations in the past. We simulated data and know the true number of divergence time from it. We are interested 
to estimate the number of divergence time from the generated data. We generated data under 5 various ways to putting prior over all
of the ways that we can assign a pair to divergence event. And then we analyzed the data under 3 models that we can model  
just we have some sense of comparion. We are intestered in how well the models have estimated the divergence time and how well 
the models have estimated what population pairs belong to which divergence time. 

Before looking at the genomic data we needed to assign prior probabilities to 3 models; how probable each model is. Dirichlet-process prior 
assigns prior to a new category with probability of 1/a+1 and an exsiting category with probability of a/a+1. The concentration parameter(a) 
makes the DP flexible because we can adjust the concentration parameter to fit our prior expectation. We can put a distribution on the 
concetration parameter and intergrate over uncertainty about the prior probabilities of the divergence models. Pitman-Yor process prior is 
generalization of Dirichlet process. There is a discount parameter(d) added in PYP and when d=0 PYP converges to DP. Drichlet has an exponential tail 
and the discount parameter gives the PYP flexibity over the tail behavior. Uniform prior is every possible divergence model is equally probable. 
Then we have specified the priors in the config files for each of these 3 models that they are comparable to each other to assess the behavior of the models. 

We used these 3 config files and then added 2 more configs (Independent and simutanous configs ) to simulate datasets. Independent config 
basically spefifies there are 10 divergence times and each population diverged indepedently. Simutanous config specifies that the populations 
diverged simutanously and there is 1 shared event. And there is another config file which ignores the constant sites(characters). A total of 6 config 
files are specified to simulate the datasets. We used these 6 config files 4 times to generate scripts and then simulated 10 datasets under each of 
these 5 models with 5000,000 characters. Then we analyzed the data under 3 models and the results show that Pitman-Yor process is doing better job handling 
most difficult scenario all 10 pairs diverging independetly because PYP is more flexible model. The coverage is more robust for Pitman-Yor process twince likely.
So we are interested to see whether this pattern holds with less data. We decided to do another simulation with 4 varying datasets 5,000, 10,000, 50,000, and 100,000. 
All the datasets simulated under 5 models and then analyzed these datasets under 3 models (DP, PYP, and SW). There are equal number of replicates for both simulations 
a total of 720 replicates for each simulation. We found that with varying datasets suggest the same Pitman-Yor process has 2 parameters and more flexible might 
have some advantages espically when it is wrong.
 

 



 
 
      

